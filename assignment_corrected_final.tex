% This LaTeX was auto-generated from MATLAB code.
% To make changes, update the MATLAB code and export to LaTeX again.

\documentclass{article}

\usepackage[utf8]{inputenc}
\usepackage[T1]{fontenc}
\usepackage{lmodern}
\usepackage{graphicx}
\usepackage{color}
\usepackage{hyperref}
\usepackage{amsmath}
\usepackage{amsfonts}
\usepackage{epstopdf}
\usepackage[table]{xcolor}
\usepackage{matlab}

\sloppy
\epstopdfsetup{outdir=./}
\graphicspath{ {./assignment_corrected_final_images/} }

\begin{document}

\begin{matlabcode}
%%all the parameters
total.mass = 52;
total.len = 1.6; %in meter
total.len_paper = 14; %in cm
total.k = total.len/total.len_paper; %scale

%Body 7 cone
body7.r=1*total.k;
body7.l = 2.2*total.k; %cone
body7.name = 'cone';
body7.cg = [(-total.len/2 + 3*body7.l/4), 0, 0]
\end{matlabcode}
\begin{matlaboutput}
body7 = 
       r: 0.1143
       l: 0.2514
    name: 'cone'
      cg: [-0.6114 0 0]

\end{matlaboutput}
\begin{matlabcode}

%body 1
body1.name = 'main_cylinder_body';
body1.r = 1*total.k;
body1.l = 10.8*total.k; 
body1.cg = [(-total.len/2 + body7.l +(body1.l)/2 ) , 0, 0]
\end{matlabcode}
\begin{matlaboutput}
body1 = 
    name: 'main_cylinder_body'
       r: 0.1143
       l: 1.2343
      cg: [0.0686 0 0]

\end{matlaboutput}
\begin{matlabcode}
body1.x1 = -total.len/2 + body7.l;
body1.x2 = total.len/2 - body1.r;

% Body 8 half spheroid
body8.name = 'hemisphere';
body8.r = 1*total.k;
body8.cg = [(total.len/2 - 5*body8.r/8), 0, 0 ]
\end{matlabcode}
\begin{matlaboutput}
body8 = 
    name: 'hemisphere'
       r: 0.1143
      cg: [0.7286 0 0]

\end{matlaboutput}
\begin{matlabcode}
body8.x1 = body1.x2;
body8.x2 = total.len/2;

%body 2 thurster 1
body2.name = 'thurster 1';
body2.r = 0.13/2;
body2.l =  2.2*total.k;
body2.cg = [(-total.len/2 + body7.l +body2.l/2), body1.r+body2.r, 0]
\end{matlabcode}
\begin{matlaboutput}
body2 = 
    name: 'thurster 1'
       r: 0.0650
       l: 0.2514
      cg: [-0.4229 0.1793 0]

\end{matlaboutput}
\begin{matlabcode}
body2.x1 = body1.x1;
body2.x2 = body2.x1 + body2.l;

%body 3 thurster 2
body3.name = 'thurster 2';
body3.r = 0.13/2;
body3.l =  2.2*total.k;
body3.cg = [(-total.len/2 + body7.l +body2.l/2), -(body1.r+body2.r), 0]
\end{matlabcode}
\begin{matlaboutput}
body3 = 
    name: 'thurster 2'
       r: 0.0650
       l: 0.2514
      cg: [-0.4229 -0.1793 0]

\end{matlaboutput}
\begin{matlabcode}
body3.x1 = body1.x1;
body3.x2 = body3.x1 + body3.l;


%body 4 rect
body4.name = 'sensor/ rect';
body4.a = 0.7 *total.k;
body4.b = 0.3 *total.k;
body4.c = 2.25 *total.k;
body4.cg = [(-total.len/4 - body4.a/2), 0, -(body1.r + body4.c/2)]
\end{matlabcode}
\begin{matlaboutput}
body4 = 
    name: 'sensor/ rect'
       a: 0.0800
       b: 0.0343
       c: 0.2571
      cg: [-0.4400 0 -0.2429]

\end{matlaboutput}
\begin{matlabcode}
body4.z1 = -(body1.r +body4.c);
body4.z2 = -(body1.r);

%body 5 cyl
body5.name = 'cyl1'; 
body5.l = 0.5*total.k;
body5.r = 0.25*total.k;
body5.cg = [3.2*total.k+body5.r, 0, -body1.r-body5.l/2]
\end{matlabcode}
\begin{matlaboutput}
body5 = 
    name: 'cyl1'
       l: 0.0571
       r: 0.0286
      cg: [0.3943 0 -0.1429]

\end{matlaboutput}
\begin{matlabcode}

%body 6 cyl2 small

body6.name = 'cyl2/small'; 
body6.l = 0.25*total.k;
body6.r = 0.25*total.k;
body6.cg = [5.5*total.k+body6.r, 0, -body1.r-body6.l/2]
\end{matlabcode}
\begin{matlaboutput}
body6 = 
    name: 'cyl2/small'
       l: 0.0286
       r: 0.0286
      cg: [0.6571 0 -0.1286]

\end{matlaboutput}
\begin{matlabcode}

\end{matlabcode}


\begin{matlabcode}
%total volume calc
body1.volume = pi*body1.l* body1.r^2;
body2.volume = pi*body2.l* body2.r^2;
body3.volume = pi*body3.l* body3.r^2;
body4.volume = body4.a* body4.b *body4.c;
body5.volume = pi*body5.l* body5.r^2;
body6.volume = pi*body6.l* body6.r^2;
body7.volume = (1/3)*pi*body7.l* body7.r^2;
body8.volume = (2/3)*pi* body8.r^3;

total.vol = body1.volume + body2.volume +body3.volume +body4.volume +body5.volume +body6.volume +body7.volume +body8.volume;
\end{matlabcode}


\begin{matlabcode}
%density cal
total.mass = 52;
total.density =  total.mass/total.vol;
\end{matlabcode}


\begin{matlabcode}
%individual mass calc
body1.m = total.density*body1.volume;
body2.m = total.density*body2.volume;
body3.m = total.density*body3.volume;
body4.m = total.density*body4.volume;
body5.m = total.density*body5.volume;
body6.m = total.density*body6.volume;
body7.m = total.density*body7.volume;
body8.m = total.density*body8.volume;

total.mass = body1.m+body2.m+body3.m+body4.m+body5.m+body6.m+body7.m+body8.m
\end{matlabcode}
\begin{matlaboutput}
total = 
         mass: 52
          len: 1.6000
    len_paper: 14
            k: 0.1143
          vol: 0.0648
      density: 802.3273

\end{matlaboutput}


\begin{matlabcode}
%inertia calc
body1.I = zeros(3,1);
body1.I(1) = (body1.m*(body1.r)^2)/2;
body1.I(2) = body1.m*(((body1.r)^2)/4 +((body1.l)^2)/12);
body1.I(3) = body1.I(2);

%body2
body2.I = zeros(3,1);
body2.I(1) = (body2.m*(body2.r)^2)/2;
body2.I(2) = body2.m*(((body2.r)^2)/4 +((body2.l)^2)/12);
body2.I(3) = body2.I(2);

%body3
body3.I = body2.I;

%body4
body4.I = zeros(3,1);
body4.I(1) = (body4.m/12)*(body4.b^2 + body4.c^2);
body4.I(2) = (body4.m/12)*(body4.a^2 + body4.c^2);
body4.I(3) = (body4.m/12)*(body4.b^2 + body4.a^2);

%body5
body5.I = zeros(3,1);
body5.I(1) = body5.m*(((body5.r)^2)/4 +((body5.l)^2)/12);
body5.I(2) = body5.I(1);
body5.I(3) = (body5.m*(body5.r)^2)/2;

%body6
body6.I = zeros(3,1);
body6.I(1) = body6.m*(((body6.r)^2)/4 +((body6.l)^2)/12);
body6.I(2) = body6.I(1);
body6.I(3) = (body6.m*(body6.r)^2)/2;

%body7
body7.I = zeros(3,1);
body7.I(1) = (3*body7.m*body7.r^2)/10;
body7.I(2) = (3*body7.m/20)*(body7.r^2+4*body7.l^2);
body7.I(3) = body7.I(2);

%body8

body8.I = zeros(3,1);
body8.I(1) = (2/5)*body8.m*body8.r^2;
body8.I(2) = (83/320)*body8.m*body8.r^2;
body8.I(3) = body8.I(2);

\end{matlabcode}


\begin{matlabcode}
%calculating mass matrix
%ig1 is the inertia matrix

body1.M = [body1.m*eye(3), zeros(3,3);
      zeros(3,3), diag(body1.I)];


%body 2
body2.M = [body2.m*eye(3), zeros(3,3);
      zeros(3,3), diag(body2.I)];


%body 3
body3.M = [body3.m*eye(3), zeros(3,3);
      zeros(3,3), diag(body3.I)];


%body 4
body4.M = [body4.m*eye(3), zeros(3,3);
      zeros(3,3), diag(body4.I)];


%body 5
body5.M = [body5.m*eye(3), zeros(3,3);
      zeros(3,3), diag(body5.I)];


%body 6
body6.M = [body6.m*eye(3), zeros(3,3);
      zeros(3,3), diag(body6.I)];


%body 7
body7.M = [body7.m*eye(3), zeros(3,3);
      zeros(3,3), diag(body7.I)];


%body 8
body8.M = [body8.m*eye(3), zeros(3,3);
      zeros(3,3), diag(body8.I)];

\end{matlabcode}


\begin{matlabcode}
%converting all the M to cg
%body 1
total.cg = [0, 0, 0];
body1.rg = body1.cg-total.cg;
body1.H = H(body1.rg);
body1.Mcg = body1.H' * body1.M * body1.H;


%body 2
body2.rg = body2.cg-total.cg;
body2.H = H(body2.rg);
body2.Mcg = body2.H' * body2.M * body2.H;


%body 3
body3.rg = body3.cg-total.cg;
body3.H = H(body3.rg);
body3.Mcg = body3.H' * body3.M * body3.H;

%body 4
body4.rg = body4.cg-total.cg;
body4.H = H(body4.rg);
body4.Mcg = body4.H' * body4.M * body4.H;


%body 5
body5.rg = body5.cg-total.cg;
body5.H = H(body5.rg);
body5.Mcg = body5.H' * body5.M * body5.H;

%body 6
body6.rg = body6.cg-total.cg;
body6.H = H(body6.rg);
body6.Mcg = body6.H' * body6.M * body6.H;

%body 7
body7.rg = body7.cg-total.cg;
body7.H = H(body7.rg);
body7.Mcg = body7.H' * body7.M * body7.H;

%body 8
body8.rg = body8.cg-total.cg;
body8.H = H(body8.rg);
body8.Mcg = body8.H' * body8.M * body8.H;

total.Mcg = body1.Mcg+body2.Mcg+body3.Mcg+body4.Mcg+body5.Mcg+body6.Mcg+body7.Mcg+body8.Mcg 
\end{matlabcode}
\begin{matlaboutput}
total = 
         mass: 52
          len: 1.6000
    len_paper: 14
            k: 0.1143
          vol: 0.0648
      density: 802.3273
           cg: [0 0 0]
          Mcg: [6x6 double]
           Ma: [6x6 double]

\end{matlaboutput}
\begin{matlabcode}
total.Mcg
\end{matlabcode}
\begin{matlaboutput}
ans = 6x6    
   52.0000         0         0         0   -0.1618         0
         0   52.0000         0    0.1618         0    0.4984
         0         0   52.0000         0   -0.4984         0
         0    0.1618         0    0.5127         0   -0.0489
   -0.1618         0   -0.4984         0    9.1490         0
         0    0.4984         0   -0.0489         0    9.2813

\end{matlaboutput}


\begin{matlabcode}
%added mass calculation
%added mass for the main body with fins
%we devide the body into 4 portions
% then calculate the added mass of each portion wrt cg
% then add them to get the total added mass of the body with fins
syms x;
%added mass calculation Fin Ma13
Ma13 = zeros(6,6);
rho = 1000;

afin = body1.r;
bfin = body1.r + 2*body2.r;

%Calculation of a22
Ca = 1;
Ar = pi*(afin)^2;
a22 = rho * Ca * Ar;
m22 = a22*body2.l; % length of thurster 

%Calculation of m33
Ca = 1-(afin/bfin)^2 + (afin/bfin)^4;
Ar = pi * bfin^2;
a33 = rho * Ca * Ar;
m33 = a33*body2.l;

%Calculation of m44
alfa = pi-(asin(2*afin*bfin/(afin^2+bfin^2)));
fa = 2*(alfa)^2 - alfa*sin(4*alfa) + 0.5*(sin(2*alfa))^2;
a44 = (rho*(afin^4)*(fa*(csc(alfa)^4)-pi^2))/(2*pi);
m44 = a44*body2.l;


fun = @(x) a33*x.^2;
m55 = integral(fun, body2.x1, body2.x2);

fun = @(x) a22*x.^2;
m66 = integral(fun, body2.x1, body2.x2);

fun = @(x) a22.*x;
m26 = integral(fun, body2.x1, body2.x2);

fun = @(x) -a33.*x;
m35 = integral(fun, body2.x1, body2.x2);

Ma13(2,2) = m22;
Ma13(3,3) = m33;
Ma13(4,4) = m44;
Ma13(5,5) = m55;
Ma13(6,6) = m66;
Ma13(2,6) = m26;
Ma13(6,2) = m26;
Ma13(3,5) = m35;
Ma13(5,3) = m35
\end{matlabcode}
\begin{matlaboutput}
Ma13 = 6x6    
         0         0         0         0         0         0
         0   10.3169         0         0         0   -4.3626
         0         0   39.0780         0   16.5244         0
         0         0         0    0.1284         0         0
         0         0   16.5244         0    7.1933         0
         0   -4.3626         0         0         0    1.8991

\end{matlaboutput}
\begin{matlabcode}

\end{matlabcode}


\begin{matlabcode}
%Added Mass Calculation of Cylinder section Ma12
%wrt cg
  
Ca = 1;
Ar = pi*body1.r^2; 
  
%Body 12
Ma12 = zeros(6,6);
a22 = rho *Ca * Ar;
m22 = a22*(body1.x2 - body2.x2); % 6+2.6 is the len of the section

a33 =a22;
m33 = m22;
m44 =0;
    
    %calcultatin m55
fun = @(x) a33*x.^2;
m55 = integral(fun, body2.x2, body1.x2);
    
    %calcultatin m66
fun = @(x) a22*x.^2;
m66 = integral(fun, body2.x2, body1.x2);
    
    %calcultatin m26
fun = @(x) a22.*x;
m26 = integral(fun, body2.x2, body1.x2);
    
    %calcultatin m35
fun = @(x) -a33.*x;
m35 = integral(fun, body2.x2, body1.x2);
    
Ma12(2,2) = m22;
Ma12(3,3) = m33;
Ma12(4,4) = m44;
Ma12(5,5) = m55;
Ma12(6,6) = m66;
Ma12(2,6) = m26;
Ma12(6,2) = m26;
Ma12(3,5) = m35;
Ma12(5,3) = m35
\end{matlabcode}
\begin{matlaboutput}
Ma12 = 6x6    
         0         0         0         0         0         0
         0   40.3296         0         0         0    7.8355
         0         0   40.3296         0   -7.8355         0
         0         0         0         0         0         0
         0         0   -7.8355         0    4.7689         0
         0    7.8355         0         0         0    4.7689

\end{matlaboutput}
\begin{matlabcode}

\end{matlabcode}


\begin{matlabcode}
%Added Mass Calculation of shpere Ma11
%wrt cg
Ma11 = zeros(6,6);
rho = 1000;
Ca = 1;
syms x;
R_sphere =  sqrt(body8.r^2 - (x-body1.x2).^2);
Ar =  pi* R_sphere^2;

a22 =  rho*Ca*Ar;
m22 = int(a22, [body8.x1,body8.x2] );

a33 = a22;
m33 = m22;
m44 = 0;

a55 =  a33;
m55 = int(a33*x^2, [body8.x1,body8.x2]);

m66 = m55;

a26 = a22;
m26 = int(a22*x, [body8.x1,body8.x2]);

m35 = -m26;

Ma11(2,2) = m22;
Ma11(3,3) = m33;
Ma11(4,4) = m44;
Ma11(5,5) = m55;
Ma11(6,6) = m66;
Ma11(2,6) = m26;
Ma11(6,2) = m26;
Ma11(3,5) = m35;
Ma11(5,3) = m35
\end{matlabcode}
\begin{matlaboutput}
Ma11 = 6x6    
         0         0         0         0         0         0
         0    3.1263         0         0         0    2.2778
         0         0    3.1263         0   -2.2778         0
         0         0         0         0         0         0
         0         0   -2.2778         0    1.6619         0
         0    2.2778         0         0         0    1.6619

\end{matlaboutput}


\begin{matlabcode}
%Added Mass Calculation of Cone Ma14
%wrt cg
Ma14 = zeros(6,6);
body7.x2 = body1.x1;
body7.y2 = -body7.r;

body7.x1 = -total.len/2;
body7.y1 = 0;

m = (body7.y2-body7.y1)/(body7.x2-body7.x1);
c = body7.y1-m*body7.x1;

line =  m*x+c;

r14 = line;

Ca = 1;
Ar =pi*r14^2;

a22 = rho*Ca*Ar;

m22 =  int(a22, body7.x1, body7.x2);

a33 =a22;
m33 = m22;
m44 =0;

%calcultatin m55
a55 = a33;
m55 = int(a33*x^2, body7.x1, body7.x2);

%calcultatin m66

m66 = m55;

%calcultatin m26
m26 =  int(a22*x, body7.x1, body7.x2);

%calcultatin m35

m35 = -m26;

Ma14(2,2) = m22;
Ma14(3,3) = m33;
Ma14(4,4) = m44;
Ma14(5,5) = m55;
Ma14(6,6) = m66;
Ma14(2,6) = m26;
Ma14(6,2) = m26;
Ma14(3,5) = m35;
Ma14(5,3) = m35;

%total main body with fin
Ma1 = Ma11 + Ma12 + Ma13 + Ma14;
Ma1(1,1) = added_mass_with_lambs_k_factor(total.len/2,body1.r)
\end{matlabcode}
\begin{matlaboutput}
Ma1 = 6x6    
    1.5691         0         0         0         0         0
         0   57.2118         0         0         0    3.6480
         0         0   85.9730         0    8.5139         0
         0         0         0    0.1284         0         0
         0         0    8.5139         0   14.9179         0
         0    3.6480         0         0         0    9.6237

\end{matlaboutput}
\begin{matlabcode}

\end{matlabcode}


\begin{matlabcode}
%Calculating X axis for thrusters (K lamb)
Ma2 = zeros(6,6);

 
Ma2(1,1) = added_mass_with_lambs_k_factor(body2.l/2,body2.r);
Ma3 = Ma2;

%body 2
Ma2cg = body2.H' * Ma2 * body2.H; %calculate the mass matrix at the cg

%body 3
Ma3cg = body3.H' * Ma3 * body3.H;

%Adding to total added mass matrix
Ma1 = Ma1 + Ma2cg + Ma3cg
\end{matlabcode}
\begin{matlaboutput}
Ma1 = 6x6    
    2.5453         0         0         0         0         0
         0   57.2118         0         0         0    3.6480
         0         0   85.9730         0    8.5139         0
         0         0         0    0.1284         0         0
         0         0    8.5139         0   14.9179         0
         0    3.6480         0         0         0    9.6551

\end{matlaboutput}


\begin{matlabcode}
% Calculate Added mass of Body 4 Ma4
Ma4 = zeros(6,6);
rho = 1000;
Ca = 1.36; %from table of sheet
aa = body4.b/2;
bb = body4.a/2;
Ar = pi * aa^2;

a11 = rho * Ca * Ar;
m11 = a11*body4.c;

aa = body4.a/2; %because the direction of motion changed
bb = body4.b/2;
Ca = 1.71; %from table
Ar = pi*aa^2;
a22 = rho *Ca * Ar;
m22 = a22*body4.c;

aa = body4.b/2; %motion direction changed
bb = body4.a/2;
a44 = 0.15*rho*pi*(bb)^4;
syms z;
fun =  a44*z^2;
m44 = int(fun,[-body4.c/2,body4.c/2]);

aa = body4.a/2;
bb = body4.b/2;
a55 = 0.15*rho*pi*(aa)^4;
fun =  a55*z^2;
m55 = int(fun,[-body4.c/2,body4.c/2]);

%approximation
m66 = 0
\end{matlabcode}
\begin{matlaboutput}
m66 = 0
\end{matlaboutput}
\begin{matlabcode}
m33 = added_mass_with_lambs_k_factor(body4.c/2, (body4.a+body4.b)/4)
\end{matlabcode}
\begin{matlaboutput}
m33 = 0.0303
\end{matlaboutput}
\begin{matlabcode}

fun =  a11*z;
m15 = int(fun,[-body4.c/2,body4.c/2]);

fun =  -a22*z;
m24 = int(fun,[-body4.c/2,body4.c/2]);

Ma4(3,3) = m33;
Ma4(2,2) = m22;
Ma4(1,1) = m11;
Ma4(4,4) = m44;
Ma4(5,5) = m55;
Ma4(6,6) = m66;
Ma4(2,4) = m24;
Ma4(4,2) = m24;
Ma4(1,5) = m15;
Ma4(5,1) = m15;


%transform to cg
%body 4
Ma4cg = body4.H' * Ma4 * body4.H
\end{matlabcode}
\begin{matlaboutput}
Ma4cg = 6x6    
    0.3229         0         0         0   -0.0784         0
         0    2.2102         0    0.5368         0   -0.9725
         0         0    0.0303         0    0.0133         0
         0    0.5368         0    0.1304         0   -0.2362
   -0.0784         0    0.0133         0    0.0249         0
         0   -0.9725         0   -0.2362         0    0.4279

\end{matlaboutput}
\begin{matlabcode}

\end{matlabcode}


\begin{matlabcode}
%Added Mass Calculation for body M5
Ma5 = zeros(6,6);
bb = body5.l;
aa = body5.r;
rho = 1000;

vr = pi*(aa^2)*bb;
%ASK-> Limit of integration
Ca = 0.62; %from table
a11 = rho*Ca*vr;
m11 = a11*2*aa; %ask this

a22 = a11;
m22 = m11;

%ASK-> If this is correct or not
m33 = m22; %ask if this or lambs which one?
%m33 = added_mass_with_lambs_k_factor(bb/2, aa)

a44 = a22;
fun = a22*z.^2;
m44 = int(fun,[-aa,aa]);

m55 = m44;

m66 = 0;

fun =  a11*z;
m15 = int(fun,[-aa,aa]);

fun = -a22.*z;
m24 = int(fun,[-aa,aa]);

Ma5(3,3) = m33;
Ma5(2,2) = m22;
Ma5(1,1) = m11;
Ma5(4,4) = m44;
Ma5(5,5) = m55;
Ma5(6,6) = m66;
Ma5(2,4) = m24;
Ma5(4,2) = m24;
Ma5(1,5) = m15;
Ma5(5,1) = m15
\end{matlabcode}
\begin{matlaboutput}
Ma5 = 6x6    
    0.0052         0         0         0         0         0
         0    0.0052         0         0         0         0
         0         0    0.0052         0         0         0
         0         0         0    0.0000         0         0
         0         0         0         0    0.0000         0
         0         0         0         0         0         0

\end{matlaboutput}
\begin{matlabcode}

%Shift of Ma5 to CG

Ma5cg = body5.H' * Ma5 * body5.H
\end{matlabcode}
\begin{matlaboutput}
Ma5cg = 6x6    
    0.0052         0         0         0   -0.0007         0
         0    0.0052         0    0.0007         0    0.0020
         0         0    0.0052         0   -0.0020         0
         0    0.0007         0    0.0001         0    0.0003
   -0.0007         0   -0.0020         0    0.0009         0
         0    0.0020         0    0.0003         0    0.0008

\end{matlaboutput}
\begin{matlabcode}

\end{matlabcode}


\begin{matlabcode}
%Added Mass Calculation for body M6
Ma6 = zeros(6,6);
bb = body6.l;
aa = body6.r;
rho = 1000;

vr = pi*(aa^2)*bb;
%ASK-> Limit of integration
Ca = 0.62; %from table
a11 = rho*Ca*vr;
m11 = a11*2*aa; %ask this

a22 = a11;
m22 = m11;

%ASK-> If this is correct or not
m33 = m22; %ask if this or lambs which one?
%m33 = added_mass_with_lambs_k_factor(bb/2, aa)

a44 = a22;
fun = a22*z.^2;
m44 = int(fun,[-aa,aa]);

m55 = m44;

m66 = 0;

fun =  a11*z;
m15 = int(fun,[-aa,aa]);

fun = -a22.*z;
m24 = int(fun,[-aa,aa]);

Ma6(3,3) = m33;
Ma6(2,2) = m22;
Ma6(1,1) = m11;
Ma6(4,4) = m44;
Ma6(5,5) = m55;
Ma6(6,6) = m66;
Ma6(2,4) = m24;
Ma6(4,2) = m24;
Ma6(1,5) = m15;
Ma6(5,1) = m15
\end{matlabcode}
\begin{matlaboutput}
Ma6 = 6x6    
    0.0026         0         0         0         0         0
         0    0.0026         0         0         0         0
         0         0    0.0026         0         0         0
         0         0         0    0.0000         0         0
         0         0         0         0    0.0000         0
         0         0         0         0         0         0

\end{matlaboutput}
\begin{matlabcode}

%Shift of Ma5 to CG

Ma6cg = body6.H' * Ma6 * body6.H
\end{matlabcode}
\begin{matlaboutput}
Ma6cg = 6x6    
    0.0026         0         0         0   -0.0003         0
         0    0.0026         0    0.0003         0    0.0017
         0         0    0.0026         0   -0.0017         0
         0    0.0003         0    0.0000         0    0.0002
   -0.0003         0   -0.0017         0    0.0012         0
         0    0.0017         0    0.0002         0    0.0011

\end{matlaboutput}
\begin{matlabcode}

\end{matlabcode}


\begin{matlabcode}
total.Ma = Ma1 +Ma4cg + Ma5cg + Ma6cg
\end{matlabcode}
\begin{matlaboutput}
total = 
         mass: 52
          len: 1.6000
    len_paper: 14
            k: 0.1143
          vol: 0.0648
      density: 802.3273
           cg: [0 0 0]
          Mcg: [6x6 double]
           Ma: [6x6 double]

\end{matlaboutput}
\begin{matlabcode}
total_added_mass = total.Ma
\end{matlabcode}
\begin{matlaboutput}
total_added_mass = 6x6    
    2.8760         0         0         0   -0.0795         0
         0   59.4298         0    0.5378         0    2.6792
         0         0   86.0110         0    8.5235         0
         0    0.5378         0    0.2589         0   -0.2357
   -0.0795         0    8.5235         0   14.9449         0
         0    2.6792         0   -0.2357         0   10.0849

\end{matlaboutput}
\begin{matlabcode}
total_mass_matrix = total.Mcg
\end{matlabcode}
\begin{matlaboutput}
total_mass_matrix = 6x6    
   52.0000         0         0         0   -0.1618         0
         0   52.0000         0    0.1618         0    0.4984
         0         0   52.0000         0   -0.4984         0
         0    0.1618         0    0.5127         0   -0.0489
   -0.1618         0   -0.4984         0    9.1490         0
         0    0.4984         0   -0.0489         0    9.2813

\end{matlaboutput}
\begin{matlabcode}
%added_mass_with_lambs_k_factor(total.len/2, body1.r)
\end{matlabcode}


\begin{matlabcode}
%drag matrix
%body 1 + body 7 + body 8
%stored in body 1

body1.D = zeros(6,6);
body1.Dy = 2*body1.r; 
body1.Dz = 2*body1.r;
body1.CD22 = 0.15; %from table
body1.CD33 = 0.15; %from table
body1.SX = pi*body1.r^2;
body1.CD11 = 0.1;

body1.D(1,1) = 0.5*rho*body1.SX * body1.CD11;
body1.D(2,2) = 0.5*rho *body1.CD22*body1.Dy*(body1.l+body7.l + body8.r);
body1.D(3,3) = 0.5*rho *body1.CD33*body1.Dz*(body1.l+body7.l + body8.r);
body1.D(5,5) = (1/64)*rho *body1.CD33*body1.Dz*(body1.l + body7.l + body8.r)^4;
body1.D(6,6) = (1/64)*rho *body1.CD22*body1.Dy*(body1.l + body7.l + body8.r)^4;
body1.D
\end{matlabcode}
\begin{matlaboutput}
ans = 6x6    
    2.0517         0         0         0         0         0
         0   27.4286         0         0         0         0
         0         0   27.4286         0         0         0
         0         0         0         0         0         0
         0         0         0         0    3.5109         0
         0         0         0         0         0    3.5109

\end{matlaboutput}
\begin{matlabcode}


%body 2
body2.D = zeros(6,6);
body2.Dy = 2*body2.r; 
body2.Dz = 2*body2.r;
body2.CD22 = 0.2;
body2.CD33 = 0.2;
body2.SX = pi*body2.r^2;
body2.CD11 = 0.1;

body2.D(1,1) = 0.5*rho*body2.SX * body2.CD11;
body2.D(2,2) = 0; %because of body shape. it does not create any significant drag compared to the main body
body2.D(3,3) = 0.5*rho *body2.CD33*body2.Dz*(body2.l);
body2.D(5,5) = (1/64)*rho *body2.CD33*body2.Dz*(body2.l )^4;
body2.D(6,6) = 0; %because of body shape. it does not create any significant drag compared to the main body
body2.D
\end{matlabcode}
\begin{matlaboutput}
ans = 6x6    
    0.6637         0         0         0         0         0
         0         0         0         0         0         0
         0         0    3.2686         0         0         0
         0         0         0         0         0         0
         0         0         0         0    0.0016         0
         0         0         0         0         0         0

\end{matlaboutput}
\begin{matlabcode}

%body 3
body3.D = body2.D;


%body 4 not complete yet
body4.D = zeros(6,6);

body4.Dx = body4.a;
body4.Dy = body4.b;

body4.CD11 = 0.7;
body4.CD22 = 0.7;

body4.Sz = body4.a * body4.b;
body4.CD33 = 1.10 + 0.02*(body4.a/body4.b + body4.b/body4.a);

body4.D(1,1) = 0.5*rho*body4.CD11 * body4.Dx;
body4.D(2,2) = 0.5*rho*body4.CD22 * body4.Dy;

body4.D(3,3) = 0; %because of body shape. it does not create any significant drag compared to the main body

body4.D(6,6) = 0;

body4.D(4,4) = (1/64)*rho*body4.CD22*body4.Dy*(body4.c)^4;
body4.D(5,5) = 0; %because of body shape. it does not create any significant drag compared to the main body
body4.D
\end{matlabcode}
\begin{matlaboutput}
ans = 6x6    
   28.0000         0         0         0         0         0
         0   12.0000         0         0         0         0
         0         0         0         0         0         0
         0         0         0    0.0016         0         0
         0         0         0         0         0         0
         0         0         0         0         0         0

\end{matlaboutput}
\begin{matlabcode}

%body 5
body5.D = zeros(6,6);

body5.Dx = 2*body5.r;
body5.Dy = 2*body5.r;

body5.CD11 = 0.3; %circular
body5.CD22 = 0.3; %circular

body5.CD33 = 0.9;

body5.D(1,1) = 0.5*rho*body5.CD11*body5.Dx;
body5.D(2,2) = 0.5*rho*body5.CD22 * body5.Dy;

body5.D(3,3) = 0; %because of body shape. it does not create any significant drag compared to the main body

body5.D(6,6) = 0; %formula

body5.D(4,4) = (1/64)*rho*body5.CD22*body5.Dy*(body5.l)^4;
body5.D(5,5) = 0; %because of body shape. it does not create any significant drag compared to the main body
body5.D
\end{matlabcode}
\begin{matlaboutput}
ans = 6x6    
    8.5714         0         0         0         0         0
         0    8.5714         0         0         0         0
         0         0         0         0         0         0
         0         0         0    0.0000         0         0
         0         0         0         0         0         0
         0         0         0         0         0         0

\end{matlaboutput}
\begin{matlabcode}

%body 6
body6.D = zeros(6,6);

body6.Dx = 2*body6.r;
body6.Dy = 2*body6.r;

body6.CD11 = 0.3; %circular
body6.CD22 = 0.3; %circular

body6.CD33 = 0.9;

body6.D(1,1) = 0.5*rho*body6.CD11*body6.Dx;
body6.D(2,2) = 0.5*rho*body6.CD22 * body6.Dy;

body6.D(3,3) = 0; %because of body shape. it does not create any significant drag compared to the main body

body6.D(6,6) = 0; %formula

body6.D(4,4) = (1/64)*rho*body6.CD22*body6.Dy*(body6.l)^4;
body6.D(5,5) = 0; %because of body shape. it does not create any significant drag compared to the main body

body6.D
\end{matlabcode}
\begin{matlaboutput}
ans = 6x6    
    8.5714         0         0         0         0         0
         0    8.5714         0         0         0         0
         0         0         0         0         0         0
         0         0         0    0.0000         0         0
         0         0         0         0         0         0
         0         0         0         0         0         0

\end{matlaboutput}
\begin{matlabcode}
%body 7
body7.D = zeros(6,6);


%body 8
body8.D = zeros(6,6);

\end{matlabcode}



\vspace{1em}
\begin{matlabcode}
function ma11 = added_mass_with_lambs_k_factor(a,b)
    e = sqrt(1-(b/a)^2);
    alpha0 = 2*((1-e^2)/e^3)*(1/2*(log((1+e)/(1-e))) - e);
    k1 = alpha0/(2-alpha0);
    mdf = (4/3)*(pi*1000*a*b^2);
    ma11 = k1*mdf;

end

function h = H(r)
    h = [eye(3) zeros(3,3);
        S(r), eye(3)];
    h = h';


end




function s= S(r)
    x = r(1);
    y = r(2);
    z = r(3);
    
    s = [0, -z, y;
        z, 0, -x;
        -y, x, 0];

end
\end{matlabcode}

\end{document}
